% !TEX TS-program = xelatex
\documentclass[12pt,oneside]{article}

\input{.house-style/preamble.tex}

\title{English Kinship Terms: From Taboo to Syntax}
\author{Brett Reynolds\orcidlink{0000-0003-0073-7195}\\[4pt]
Humber Polytechnic \& University of Toronto}
\date{\today}

\begin{document}

\maketitle

\begin{abstract}
English-speaking children rarely address parents by first name~-- a social prohibition encoding power and solidarity. Separately, English kinship terms like \mention{Mom} and \mention{Dad} permit bare singular use (\mention{Mom called}) where other count nouns don't (\ungram{\mention{Neighbor called}}). Nobody has connected these facts. I argue that the naming taboo causes the proper-name-like syntax: high-frequency use of kinship terms in vocative position~-- forced by the prohibition on first names~-- produces grammaticalization toward strong proper name status. The CGEL framework provides the analytical apparatus: kinship terms can be used with proper-name status (bare NP; family-domain identifiable reference) or retain their relational meaning (with determiner). This functional split reflects positional grammaticalization, not polysemy. The taboo is the engine; the syntax is the sediment.
\end{abstract}

\textbf{Keywords:} kinship terms, naming taboo, proper nouns, grammaticalization, address forms

\section{Introduction}\label{sec:intro}

In \emph{The Parent Trap} (1998), the emotional center of the reunion isn't a revelation of biology; it's a restoration of a word. After eleven years engineered around separation~-- one parent per child, one child per parent~-- the father finds himself face-to-face with the daughter who has grown up without him. The dialogue turns on something more miniature and more brutal: what you're allowed to \emph{call} someone.

The father, Nick, reaches for the point that matters: \enquote{So let me see if I get this~-- you missed being able to call me Dad?} Annie, the daughter, answers just as plainly: \enquote{Yeah. I really have, Dad.}

The exchange is not about identifying the referent; both speakers know exactly who's being talked to. The ache is about a social relation that English encodes~-- and polices~-- through address, where \mention{Dad} isn't descriptive content but a licensed stance of affiliation, entitlement, and obligation.

That licensing is tight. A child who replaces \mention{Dad} with \mention{Nick} isn't merely \enquote{using a different label}; they're staging distance. And when the kinship term returns, it doesn't just report reconciliation~-- it performs it. The paper that follows treats that everyday naming taboo as a grammatical force: one reason English \mention{Mom} and \mention{Dad} behave, in key syntactic positions, less like ordinary common-noun NPs and more like strong proper names.

\subsection{Two facts that should be connected}

Two facts about English kinship terms have been documented extensively. Neither literature cites the other. No one has proposed that they're causally linked.

\textbf{Fact 1 (Sociolinguistic):} In mainstream Anglophone norms, children rarely address parents by first name. This is a social prohibition encoding power and solidarity \citep{BrownGilman1960,Dickey1997}. The asymmetry is robust: parents use children's first names freely; children use kinship terms. Violating the taboo~-- calling your mother \mention{Sarah} at the dinner table~-- is face-threatening, marked, and culturally dispreferred in unmarked contexts.

\textbf{Fact 2 (Syntactic):} English kinship terms permit bare singular use where other count nouns don't:

\ea\label{ex:baresingular}
    \ea[]{{\mention{Mom's here.}} \hfill (grammatical)}
    \ex[?]{{\mention{Teacher's here.}} \hfill (marginal without context)}
    \ex[*]{{\mention{Neighbor's here.}} \hfill (ungrammatical)}
    \z
\z

When used this way, \mention{Mom} functions semantically as a \term{proper name}~-- picking out a unique, identifiable individual. This semantic function is borne out syntactically: \mention{Mom} appears as a bare NP~-- what \textcite{Huddleston2002} call a \term{strong proper name}. With a determiner~-- \mention{my mom}, \mention{the mom in question}~-- the same lexical item retains its relational meaning. The bare singular pattern is remarkable because English generally resists bare singular count nouns in argument position (\ungram{\mention{I bought table}})~-- with limited exceptions in institutional frames (\mention{go to school}, \mention{in hospital}).

\textbf{The gap:} Nobody has connected these facts. The sociolinguistics literature documents the address asymmetry through power/solidarity frameworks. The syntax literature documents the determiner alternation and bare singular behavior. Neither cites the other; neither proposes a causal mechanism.

\subsection{The proposed mechanism}

This paper fills that gap. The argument is simple: the social prohibition on first-name use forces high-frequency use of kinship terms in vocative and address position. That frequency~-- sustained across childhood and into adulthood~-- produces exactly the conditions under which grammaticalization toward proper-noun-like syntax occurs.

The causal pathway has four steps:

First, \emph{social prohibition}: children can't use parents' first names. The taboo is enforced by correction, by awkwardness, by the visceral wrongness of hearing a peer say \enquote{Hey, Nick} to their father.

Second, \emph{address function}: the prohibition forces kinship terms into the slot normally occupied by proper names. \mention{Mom} and \mention{Dad} become the primary vocative forms~-- the words a child uses to get a parent's attention, to address them directly, to refer to them in third-person conversation with other family members.

Third, \emph{high frequency}: this isn't occasional use. Children say \mention{Mom} and \mention{Dad} with extreme frequency over childhood~-- even 30--50 uses per day yields 200,000--300,000 tokens over 15--20 years. Each use is in a slot prototypically occupied by proper names.

Fourth, \emph{grammaticalization}: repetition in a name-like slot produces entrenchment. The form becomes highly entrenched as a name-like address form with the distributional privileges associated with proper-name status~-- it appears without a determiner, it resists modification in ways that common nouns don't~-- while acquiring the semantic function of a proper name: identifiable reference within the family domain.

Grammar, on this view, is \enquote{sedimented behavior} \citep{bybee2010}, not a set of abstract rules that \enquote{allow} exceptions. \mention{Mom} isn't a common noun that the grammar happens to permit in a bare-singular slot. It's a form that has acquired proper-noun distribution through the pressure of its use.

\subsection{The CGEL framework}

The analytical apparatus comes from \emph{The Cambridge Grammar of the English Language} \citep{Huddleston2002}, which maintains strict category--function distinctions that earlier treatments conflate.

\begin{table}[h]
\centering
\begin{tabular}{lll}
\toprule
Concept & Definition & Example \\
\midrule
Proper noun & Word-level: noun specialized for naming & \mention{John}, \mention{Clinton} \\
Proper name & Phrase-level: NP adopted as name & \mention{the United Kingdom} \\
Strong proper name & Proper name without article & \mention{Paris}, \mention{Mom} \\
Weak proper name & Proper name requiring \mention{the} & \mention{the Thames} \\
Bare singular & Count noun without determiner & Normally ungrammatical \\
\bottomrule
\end{tabular}
\caption{Key terminology from CGEL}
\label{tab:cgel}
\end{table}

Kinship terms show a functional split:

\ea\label{ex:split}
    \ea Proper-name semantics, proper-noun distribution: \mention{Mom called.}
    \ex Relational semantics, default distribution with determiner: \mention{My mom called.}
    \z
\z

CGEL itself notes this split in its treatment of vocatives (\S5.20.5): kin terms are listed alongside personal names as NPs that can serve vocative function, and the text explicitly observes that \enquote{the kin terms include those that can be used with the status of proper names} \citep[522]{Huddleston2002}. \mention{Mum}, \mention{Mom}, and \mention{Mummy} are given as examples. Other kin terms~-- \mention{son}, \mention{cousin}~-- are \enquote{hardly possible as proper names,} showing that the category is graded rather than uniform. CGEL provides the descriptive distinction; the present paper proposes an explanatory pathway.

This isn't polysemy in the usual sense~-- two separate lexemes that happen to share a form. Several diagnostics favor a constructional/status analysis over a two-lexeme account: the use is gradient across kinship terms (not categorical), depends on discourse domain (family vs outside), interacts predictably with modification and possessives, shows cross-speaker variability, and exhibits bridging through address contexts. The grammaticalization story explains \emph{why} kinship terms developed this functional split: the taboo forces them into proper-name function so frequently that they've acquired the syntactic properties associated with that function.

\subsection{Relation to deitality and definiteness}

In a companion analysis, I distinguish \term{deitality}~-- a morphosyntactic property cluster~-- from \term{definiteness}~-- a semantic one. Deital determiners (\mention{the}, \mention{this}, \mention{my}) show distributional restrictions: they resist existential pivots, license partitive complements, host identificational constructions. Definiteness is interpretive: identifiability, uniqueness, anaphoric recoverability.

Proper names occupy a distinctive position in this framework: they're \emph{semantically definite} (they refer to unique, identifiable individuals) but \emph{morphosyntactically non-deital} (they appear without determiners and don't show the distributional restrictions of deital NPs). English kinship terms, when functioning as proper names, pattern exactly this way. They're definite in meaning but bare in form~-- grammaticalized fossils of the taboo that shaped their use.

This explains why \mention{Mom called} behaves more like \mention{Paris welcomed visitors} than like \mention{My mother called}. Both \mention{Mom} and \mention{Paris} are strong proper names: definite in interpretation, non-deital in morphosyntax. The kinship term has been pulled toward proper-name status by the social pressure of address.

\subsection{What follows}

The rest of the paper proceeds as follows. Section~\ref{sec:socio} reviews the sociolinguistic literature on address forms and the naming taboo. Section~\ref{sec:syntax} reviews the syntactic literature on proper nouns, bare nominals, and the CGEL apparatus. Section~\ref{sec:mechanism} presents the core argument: the grammaticalization pathway from taboo to syntax. Section~\ref{sec:crossling} addresses the cross-linguistic dimension. Section~\ref{sec:objections} considers objections and alternative explanations. Section~\ref{sec:conclusion} concludes.

%% PLACEHOLDER SECTIONS %%

\section{The naming taboo}\label{sec:socio}

The literature on address forms has documented a robust asymmetry in English-speaking families: parents address children by first name; children address parents by kinship term. This isn't a quirk of individual families. It's a social rule, enforced by correction and marked by discomfort when violated.

\subsection{Power and solidarity}

The foundational framework comes from \textcite{BrownGilman1960}, who analyzed the pronouns of power and solidarity in European languages. Their T/V distinction~-- the choice between familiar \mention{tu} forms and formal \mention{vous} forms~-- revealed that pronoun selection encodes social relationships along two dimensions: \term{power} (asymmetric, hierarchical) and \term{solidarity} (symmetric, affiliative). Superior-to-inferior relations license non-reciprocal address; equals negotiate mutual forms.

The framework extends readily to names and titles. \textcite{BrownFord1961} applied the power/solidarity model to American English address, documenting a system where first-name (FN) and title-last-name (TLN) usage follows predictable patterns. Superiors give first names and receive titles; intimates exchange first names; formal equals exchange titles. The key finding for our purposes is that address form asymmetry reliably encodes power differentials.

\subsection{The parent-child asymmetry}

Within the family, the parent-child dyad exhibits a paradigmatic power asymmetry. Parents freely use children's first names: \mention{Sarah, time for dinner}. Children rarely reciprocate: \ungram{\mention{Sarah, when's dinner?}} (addressed to their mother named Sarah). Instead, children use kinship terms: \mention{Mom, when's dinner?}

This asymmetry persists into adulthood. Even when children are themselves parents, even when they have adult relationships with their aging parents, first-name address remains marked. The taboo isn't about childhood dependency; it's about a social relation encoded in language and maintained across the lifespan.

\textcite{Dickey1997} provides a systematic treatment of address forms and terms of reference, documenting the forms available in English and their sociopragmatic conditioning. The kinship-term-for-parent pattern is noted as near-categorical in non-marked contexts. What makes it a \term{taboo} rather than a mere preference is the affective response to violation: discomfort, face-threat, sometimes correction.

\subsection{Politeness and face}

The naming taboo is best understood through politeness theory. \textcite{BrownLevinson1987} analyze linguistic politeness as the management of \term{face}~-- the public self-image that interactants claim and mutually support. Address violations threaten face. A child who calls their parent by first name is, in politeness-theoretic terms, performing a face-threatening act: claiming solidarity equivalence where hierarchy obtains.

This face-threat explains the affective charge of the taboo. The \enquote{wrongness} of hearing a peer say \enquote{Hey, Nick} to their father isn't merely unconventional. It indexes a stance~-- claimed equality, refused deference~-- that violates expectations about parent-child relations. The discomfort is social, not merely linguistic.

\subsection{Implications for frequency}

What matters for the grammatical story is the consequence of the taboo for usage frequency. Because children can't use parents' first names, they must use kinship terms. This isn't a free choice among alternatives; it's a constrained selection with the constraint enforced by social pressure.

The result is massive frequency differential. A child says \mention{Mom} or \mention{Dad} thousands of times per year, every year, from first words through adulthood. Each use occurs in a slot prototypically occupied by proper names~-- the vocative position, the subject of address verbs, the referential slot in family conversations. The taboo doesn't just prohibit one form; it channels usage toward another, creating the conditions for grammaticalization.

\section{Kinship terms with proper-name status}\label{sec:syntax}

The syntactic facts about kinship terms have been described, but their significance hasn't been connected to the sociolinguistic facts reviewed above. This section provides the precise analytical apparatus for characterizing what's grammatically distinctive about \mention{Mom} and \mention{Dad}.

\subsection{The CGEL framework}

\emph{The Cambridge Grammar of the English Language} \citep{Huddleston2002} distinguishes carefully between lexical category and naming function. A \term{proper noun} is a lexical category~-- a subcategory of nouns specialized for naming. A \term{proper name} is an expression conventionally adopted to denote a unique entity. The two can come apart.

Consider \mention{the United Kingdom}. The noun \mention{kingdom} is a common noun, but the NP \mention{the United Kingdom} functions as a proper name: it refers to a unique political entity. The lexical category is common, the semantic function proper.

CGEL further distinguishes \term{strong proper names}~-- those that appear without an article (\mention{Paris}, \mention{Mom})~-- from \term{weak proper names}~-- those that require \mention{the} (\mention{the Thames}, \mention{the Hague}). The terminology requires care: \term{proper name} is a semantic term, where \term{strong proper name} is a bit of a hybrid. A \term{strong proper name} is a bare NP (syntax) functioning as a proper name (semantics). It's definite in reference (unique, identifiable) but appears bare, without the determiner that typically marks definiteness~-- and that is usually syntactically required in other contexts.

\subsection{The bare singular pattern}

English virtually prohibits bare singular count nouns in argument position. You can say \mention{Dogs bark} (bare plural, generic) and \mention{Water flows} (bare mass), but not \ungram{\mention{Dog barks}} or \ungram{\mention{Table broke}}.

The exceptions reveal the pattern. Proper nouns resist determiners: \mention{Paris is beautiful}, not \ungram{\mention{The Paris is beautiful}}. Bare singulars in argument position are the province of proper nouns: forms that have escaped the determiner requirement because they function as strong proper names.

Kinship terms enter this picture precisely when they function as proper names. In \mention{Mom called}, the kinship term appears bare and refers uniquely~-- exactly like \mention{John called}. The determiner is absent not because grammar has made a special exception but because the form functions as a strong proper name.

This parallel extends to pragmatics. \mention{John called} is felicitous only when both parties know which John is meant; if two Johns are in play, disambiguation is called for (\mention{John from work}, \mention{John Smith}). Similarly, bare \mention{Mom} is the norm within the family, where uniqueness is guaranteed. Outside family discourse, where everyone has a mother, \mention{my mom} restores the determiner.

\subsection{Gradience across kinship terms}

Not all kinship terms behave alike. CGEL notes this explicitly in its discussion of vocatives (\S5.20.5): \enquote{the kin terms include those that can be used with the status of proper names}, but the category is graded.

\ea\label{ex:gradient}
    \ea[]{Fully licensed: \mention{Mom}, \mention{Dad}, \mention{Grandma}, \mention{Grandpa}}
    \ex[?]{Marginal: \mention{Auntie}, \mention{Uncle}}
    \ex[*]{Unlicensed: \mention{Brother}, \mention{Sister}, \mention{Aunt}, \mention{Cousin}, \mention{Nephew}}
    \z
\z

The hierarchy doesn't track raw contact frequency~-- children interact with siblings constantly. It tracks \emph{taboo-driven} frequency: the rate at which social prohibition forces the kinship term into name-position. For parents and grandparents, the naming taboo blocks first-name use; the kinship term is the only option. For siblings, cousins, and aunts/uncles, first names are available, so the kinship term competes and loses frequency in name-position.

\mention{Auntie} and \mention{Uncle} occupy an intermediate position. Unlike most English honorifics, which combine with surnames (\mention{Dr.\ Smith}, \mention{Mr.\ Jones}), these combine with first names (\mention{Auntie Sarah}, \mention{Uncle John}). This quasi-titular use gives them name-position exposure that siblings lack. \mention{Brother} and \mention{Sister} are unlicensed in ordinary family discourse but can function as titles in religious contexts (\mention{Brother John}, \mention{Sister Mary}). In Quaker communities, for instance, bare \mention{Sister} appears in subject position: \enquote{Sister said we could make no such engagements} \citep[qtd.\ in][]{Hodgkin1927}. The mechanism generalizes beyond nuclear-family kinship.

\subsection{Vocatives and subjects}

CGEL treats \term{vocative} as a function, not a case~-- English has no vocative case, and CGEL uses the term exclusively for the function (\S5.20.5). Vocative NPs are not dependents of the verb; they can stand alone without ellipsis and are best regarded as \enquote{a kind of interpolation~-- one that can appear, like certain adjuncts, in front, central, or end position}. What's distinctive about kinship terms is that they pattern with personal names in this function: \mention{Mom, come here} parallels \mention{Sarah, come here}. Both are direct address forms; both trigger the same intonational contour; both resist modification in the same ways.

\textcite{Hill2022} adds a structural account: vocative phrases have dedicated syntactic positions, and forms that occupy those positions share grammatical properties. The vocative slot is a grammaticalization highway~-- a context where high-frequency forms acquire special privileges.

This matters for the causal story. If kinship terms are used primarily in vocative position (because the naming taboo blocks first-name use), and if vocative position selects for proper-name-like properties, then the taboo creates the grammaticalization pressure. The syntax follows from the usage pattern.

\section{From taboo to syntax}\label{sec:mechanism}

The previous sections have established two facts and the analytical apparatus for connecting them. The sociolinguistic fact: children can't use parents' first names. The syntactic fact: kinship terms like \mention{Mom} exhibit proper-noun distribution. This section presents the mechanism that links them.

\subsection{The frequency argument}

The heart of the proposal is frequency. \textcite{bybee2010} argues that grammatical patterns emerge from usage: forms that occur frequently in particular slots become entrenched in memory, stored as units, and eventually acquire the distributional properties associated with those slots. Grammar isn't a set of abstract rules that happens to permit certain forms; grammar \emph{is} the sedimented residue of repeated use.

The naming taboo creates massive frequency asymmetry. A child who can't say \mention{Nick} says \mention{Dad} instead~-- thousands of times per year, from first words through adolescence and beyond. These uses aren't scattered across positions. They're concentrated in slots prototypically occupied by proper names: vocatives (\mention{Dad, look!}), subjects of attention-getting verbs (\mention{Dad says...}), referential uses in family discourse (\mention{Dad thinks...}).

This concentration matters. \textcite{bybee2006} shows that frequency in a particular construction produces entrenchment in that construction. A form used overwhelmingly in name position comes to be \emph{stored as} a name~-- not analyzed compositionally as \enquote{the parent who is my father} but retrieved holistically as a unit with the syntactic properties of proper nouns.

\subsection{The grammaticalization pathway}

Following \textcite{Traugott1993}, we can map a grammaticalization trajectory for kinship terms:

\ea\label{ex:pathway}
    \ea Stage 1 (Source construction): relational NP headed by a common noun with genitive determinative: \mention{my mom}
    \ex Stage 2 (Bridging context): vocative function (determinerless by default): \mention{Mom}, can you...?
    \ex Stage 3 (Extension / reanalysis): bare NP in argument position within the family domain: Tell \mention{Mom} I'm late.
    \ex Stage 4 (Target construction): strong proper name (proper-name status; non-deital, bare NP in argument position): \mention{Mom} called.
    \z
\z

The bridging context is crucial. Vocative position licenses bare NPs~-- you can say \mention{Waiter!} without a determiner. But vocative frequency alone doesn't explain proper-name syntax. \mention{Waiter} doesn't license bare singular subjects with a definite, individuated reading in ordinary conversation: \ungram{\mention{Waiter called}} (with intended meaning "the waiter we both know").

What distinguishes kinship terms is the \emph{sustained frequency} across both vocative and referential uses. The taboo ensures that \mention{Mom} isn't just an occasional vocative; it's the \emph{default name} for referring to one's mother. This pushes the form beyond vocative bridging into full proper-name status.

\subsection{Evidence of ongoing grammaticalization}

The process isn't complete, and evidence of gradience supports the usage-based account. In contemporary mainstream family discourse:

\begin{itemize}
\item \mention{Mom} and \mention{Dad}~-- the highest-frequency kinship terms~-- show the most robust proper-name syntax.
\item \mention{Grandma} and \mention{Grandpa} follow closely: high address frequency in families where grandparents are present.
\item \mention{Mother} and \mention{Father}, more formal and lower-frequency, are attested as bare singulars but register-sensitive (diary style, conservative familylects, religious contexts).
\item \mention{Uncle} and \mention{Auntie} are marginal: \mention{\textsuperscript{?}Uncle said so} requires supportive context or a family-internal convention.
\item \mention{Cousin}, \mention{Nephew}, \mention{Niece}~-- rarely used in address~-- don't license bare singulars.
\end{itemize}

This hierarchy is consistent with address frequency, though detailed corpus work remains to be done. It's not an arbitrary list; it's a prediction of the mechanism. Terms pushed into name position by high-frequency address should acquire proper-name syntax; terms with lower address frequency shouldn't.

\subsection{The taboo as causal engine}

The naming taboo isn't incidental to this story. It's the engine that drives the frequency asymmetry. Without the taboo~-- in a culture where children freely used parents' first names~-- kinship terms would have lower address frequency. They might still be relational nouns (as they are in \mention{my mom}), but they wouldn't be pushed toward proper-name status.

This proposal makes a prediction: in speech communities where the naming taboo is weaker or absent, kinship terms should show less proper-noun-like syntax. They should require determiners more often; bare singular subjects should be marked or ungrammatical. A parallel prediction applies to honorific use: where kinship terms function as titles or honorifics (as with \mention{Auntie Sarah}, \mention{Uncle John}, or religious \mention{Brother}/\mention{Sister}), they should show stronger proper-name properties than when used purely relationally. Section~\ref{sec:crossling} explores these predictions.

\subsection{Grammar as sedimented behavior}

The theoretical upshot is that proper-noun syntax for kinship terms isn't a stipulated exception to general rules about determiners. It's a predictable consequence of how frequently, in what positions, these forms are used.

\textcite{bybee2010} calls grammar \enquote{sedimented behavior}: the patterns that emerge when repeated interactions leave traces in cognitive organization. The bare singular licensing of \mention{Mom} is sediment from millions of uses in vocative and referential name position. The taboo created the pressure; frequency created the entrenchment; entrenchment created the syntax.

This framing unifies the sociolinguistic and syntactic facts. The naming taboo and the bare singular pattern aren't coincidentally co-occurring. They're causally connected: the taboo produces the usage pattern that produces the syntax.

\section{Cross-linguistic perspective}\label{sec:crossling}

The English naming taboo is culturally robust but linguistically moderate. Cross-linguistic comparison suggests that the mechanism proposed here should operate with varying force in different languages, depending on the strength of the naming taboo and the structure of the kinship-term system. The relevant predictions concern \emph{name-like status effects}~-- diagnostics appropriate to each language's morphosyntax, not simply determiner omission.

\subsection{Stronger taboos: Chinese and Vietnamese}

In Mandarin Chinese, the naming taboo extends far beyond parent-child address. The traditional system of \term{bìhuì} (\enquote{name avoidance}) prohibited speaking or writing the personal names of emperors, ancestors, and senior family members \citep{Adamek2012}. Children not only can't address parents by first name; historically, they couldn't use any character appearing in a parent's name. The kinship-term system is correspondingly elaborate: distinct terms differentiate maternal from paternal relatives, older from younger siblings, and generation levels with precision exceeding English.

Vietnamese exhibits similar complexity. Kinship terms function as the primary pronoun system, with selection depending on relative age and social relationship \citep{Luong1990,Tang2007}. Usage is highly context-, region-, and relation-sensitive. These languages are not places to test determiner omission in the English sense, but they illustrate what strong address-taboo systems look like: pervasive kinship-term use in slots that other languages fill with pronouns or names.

These facts generate predictions. If the naming taboo drives proper-name status through high-frequency use in name position, then languages with stronger taboos should show correspondingly robust name-like properties for kinship terms~-- tested through diagnostics appropriate to each language (prosody, resistance to classifiers/modifiers, constraints on co-occurring pronouns, address/reference asymmetries). Detailed syntactic analysis remains beyond the present paper's scope.

\subsection{Weaker taboos}

The converse prediction is equally important. In speech communities where the naming taboo is weaker~-- where children sometimes use parents' first names, or where the taboo applies only in formal contexts~-- kinship terms should show less name-like syntax.

Scandinavian languages would provide a useful test case: anecdotally, first-name address to parents is less marked in some families and regions than in mainstream Anglophone norms. If the mechanism is correct, kinship terms in such communities should pattern more like ordinary relational nouns~-- requiring determiners more often, resisting bare singular subjects. Systematic sociolinguistic work on such communities would provide a direct test.

\subsection{Kinship terms as honorifics}

The honorific prediction is testable within English and cross-linguistically. In English, kinship terms used as quasi-honorifics (\mention{Auntie Sarah}, \mention{Uncle John}) show intermediate proper-name properties, while those used as full titles in religious contexts (\mention{Brother John}, \mention{Sister Mary}) show robust proper-name licensing, including bare subjects (\mention{Sister said we could make no such engagements}; \citealt{Hodgkin1927}).

Vietnamese presents a striking case \citep{Luong1990}. Kinship terms function as the primary pronoun system, with speakers selecting terms based on relative age and social relationship. When addressing a woman older than one's mother, one uses \mention{bác} (\enquote{aunt, older than parent}); when addressing a younger woman in a teacher role, one uses \mention{cô} (\enquote{aunt, younger than parent}). These terms extend beyond biological kin to strangers, functioning as honorifics. The prediction is that such terms should show strong name-like properties in their distribution and interaction with other referring expressions.


\subsection{Methodological note}

Cross-linguistic work on kinship semantics is extensive; cross-linguistic work on kinship-term \emph{syntax} is thinner. The prediction here concerns the interaction between address-form taboos and determiner syntax~-- a narrower question than the semantic structure of kinship systems. Testing the prediction requires careful attention to bare nominal licensing in each language, controlling for the language's more general patterns of article use (or absence).

The present paper restricts itself to English, where both the sociolinguistic and syntactic facts are well documented. The cross-linguistic predictions are offered as consequences of the mechanism, not as confirmed findings.

\section{Objections and alternatives}\label{sec:objections}

The proposal connects sociolinguistic facts to syntactic patterns through a frequency-based mechanism. Several objections and alternative explanations deserve consideration.

\subsection{Objection: Vocatives don't generalize}

\textbf{Objection:} Many nouns occur in vocative position without acquiring proper-name syntax. \mention{Waiter!} is a perfectly good vocative, but \ungram{\mention{Waiter called}} (with definite, individuated reading) is ungrammatical. Why should kinship terms be different?

\textbf{Reply:} The key distinction is between \emph{role-anchored definites} (\mention{Waiter}, \mention{Doctor}, \mention{Boss}) and \emph{conventionalized personal name substitutes} (\mention{Mom}, \mention{Dad}). Role nouns can be used vocatively and may license bare singulars in domain-bound contexts (\mention{Boss is here} in workplace narrative), but they don't become the \emph{default cross-construction name} for referring to a specific individual.

Of course, there's no taboo forcing \mention{Waiter} into name position. A customer uses \mention{Waiter!} because they don't know the server's name, not because using it would be socially prohibited. If they learned the name, they could use it freely. The kinship case is different: the child \emph{knows} the parent's name but \emph{can't use it}. This prohibition, sustained over years, forces the kinship term into name position persistently enough to produce entrenchment.

\subsection{Objection: The semantics explains it}

\textbf{Objection:} Kinship terms permit bare singular use because they denote unique referents within a family. The proper-name syntax follows from the semantics, not from frequency or the taboo.

\textbf{Reply:} The semantic uniqueness story faces empirical problems. First, the gradience: \mention{Cousin} also denotes a unique individual in many families (an only cousin), but \ungram{\mention{Cousin called}} is degraded where \mention{Mom called} isn't. Second, the asymmetry: the same lexical item~-- \mention{mom}~-- functions as a strong proper name when bare (\mention{Mom called}) but as a common noun with determiner (\mention{my mom called}). If uniqueness licensed proper-name syntax, we'd expect consistency.

The frequency mechanism explains the gradience: terms used more frequently in address show stronger proper-name properties. The semantic account doesn't predict this pattern.

\subsection{Objection: What about sir and ma'am?}

\textbf{Objection:} Address forms like \mention{sir} and \mention{ma'am} also permit bare use. Does the analysis generalize?

\textbf{Reply:} Yes, appropriately. \mention{Sir} and \mention{ma'am} are high-frequency address forms with their own grammaticalization histories. They show some proper-name-like properties: \mention{\textsuperscript{?}Sir called} is marginal but better than \ungram{\mention{Waiter called}}. The mechanism predicts exactly this: forms used frequently in address position should show intermediate proper-name properties.

The kinship story isn't unique to kinship. It's an instance of a general mechanism~-- frequency in name position produces proper-name syntax~-- applied to a case where social prohibition creates the requisite frequency.

\subsection{Alternative: Inherent lexical specification}

An alternative account would treat proper-name syntax as an inherent lexical property of certain kinship terms. On this view, \mention{mom} has two lexical entries: a common noun (relational, requiring determiner) and a proper noun (name-like, bare).

This analysis captures the synchronic facts but leaves them unexplained. Why should \mention{mom} have developed a lexicalized proper-name use, and why should that use be gradient across kinship terms? Why is bare \mention{Mom} fully licensed, bare \mention{\textsuperscript{?}Uncle} marginal, and bare \mention{\textsuperscript{*}Cousin} ungrammatical? The lexical-specification account treats as brute fact what the frequency mechanism derives from usage patterns.

Moreover, the lexical account doesn't connect to the sociolinguistic facts at all. It treats the naming taboo and the bare singular pattern as coincidental. The frequency mechanism treats them as causally connected~-- and that connection is the paper's core contribution.

\section{Conclusion}\label{sec:conclusion}

This paper has connected two facts that existing literature treats as unrelated: the English naming taboo (children can't address parents by first name) and the proper-name syntax of kinship terms (\mention{Mom called} is grammatical where \ungram{\mention{Neighbor called}} isn't). The connection is causal. The taboo creates the frequency; the frequency creates the syntax.

The mechanism is grammaticalization through usage. When the naming taboo forces children to use kinship terms instead of first names, those terms get used thousands of times in slots prototypically occupied by proper names. Repetition produces entrenchment; entrenchment produces proper-name distribution. The bare singular licensing of \mention{Mom} is sedimented behavior~-- the grammatical residue of a social prohibition.

Three contributions stand out. First, the paper bridges two literatures that have ignored each other. Sociolinguists studying address forms haven't examined determiner syntax; syntacticians studying bare nominals haven't considered the politics of the family. The connection proposed here invites further work at this interface.

Second, the paper provides an explanatory mechanism, not just a description. Other treatments note that kinship terms can function as proper names; this paper explains \emph{why}. The naming taboo is the engine; the frequency is the transmission; the syntax is the sediment.

Third, the paper offers falsifiable predictions. Across kinship terms, gradience in proper-name syntax should track address frequency. Across languages, the strength of the naming taboo should predict the robustness of proper-name properties for kinship terms. Within communities, weakening of the taboo should correlate with weakening of proper-name syntax.

Return to \emph{The Parent Trap}. When Annie tells her father \enquote{I really have, Dad}, the emotional force isn't in the semantics. Both speakers already know who's being referred to. The force is in the restoration of a license: the right to use that word, in that slot, with that person. The naming taboo made \mention{Dad} the only available term; a decade of absence didn't change the form's status as a proper name. It still appeared bare, still referred uniquely, still carried the grammatical signature of frequency-driven grammaticalization.

The syntax is the sediment. And the sediment preserves the social relation that created it.

\printbibliography

\end{document}
