% !TEX TS-program = xelatex
\documentclass[12pt,oneside]{article}

\input{.house-style/preamble.tex}

\title{English Kinship Terms: From Taboo to Syntax}
\author{Brett Reynolds\orcidlink{0000-0003-0073-7195}\\[4pt]
Humber Polytechnic \& University of Toronto}
\date{\today}

\begin{document}

\maketitle

\begin{abstract}
English-speaking children virtually never address parents by first name~-- a social prohibition encoding power and solidarity. Separately, English kinship terms like \mention{Mom} and \mention{Dad} permit bare singular use (\mention{Mom called}) where other count nouns don't (\ungram{\mention{Neighbor called}}). Nobody has connected these facts. I argue that the naming taboo causes the proper-noun-like syntax: high-frequency use of kinship terms in vocative position~-- forced by the prohibition on first names~-- produces grammaticalization toward strong proper name status. The CGEL framework provides the analytical apparatus: kinship terms function as strong proper names (bare, unique reference) or as common nouns (with determiner, relational meaning). This functional split reflects positional grammaticalization, not polysemy. The taboo is the engine; the syntax is the sediment.
\end{abstract}

\textbf{Keywords:} kinship terms, naming taboo, proper nouns, grammaticalization, address forms

\section{Introduction}\label{sec:intro}

In \emph{The Parent Trap} (1998), the emotional center of the reunion isn't a revelation of biology; it's a restoration of a word. After eleven years engineered around separation~-- one parent per child, one child per parent~-- the father finds himself face-to-face with the daughter who has grown up without him. The dialogue turns on something more miniature and more brutal: what you're allowed to \emph{call} someone.

Nick reaches for the point that matters: \enquote{So let me see if I get this~-- you missed being able to call me Dad?} Annie answers just as plainly: \enquote{Yeah. I really have, Dad.}

Nothing in that exchange is about reference; both speakers know exactly who's being talked to. The ache is about a social relation that English encodes~-- and polices~-- through address, where \mention{Dad} isn't descriptive content but a licensed stance of affiliation, entitlement, and obligation.

That licensing is tight. A child who replaces \mention{Dad} with \mention{Nick} isn't merely \enquote{using a different label}; they're staging distance. And when the kinship term returns, it doesn't just report reconciliation~-- it performs it. The paper that follows treats that everyday naming taboo as a grammatical force: one reason English \mention{Mom} and \mention{Dad} behave, in key syntactic positions, less like common nouns and more like proper names.

\subsection{Two facts that should be connected}

Two facts about English kinship terms have been documented extensively. Neither literature cites the other. No one has proposed that they're causally linked.

\textbf{Fact 1 (Sociolinguistic):} English-speaking children virtually never address parents by first name. This is a social prohibition encoding power and solidarity \citep{BrownGilman1960,Dickey1997}. The asymmetry is stark: parents use children's first names freely; children use kinship terms. Violating the taboo~-- calling your mother \mention{Sarah} at the dinner table~-- is face-threatening, marked, and culturally dispreferred across social classes.

\textbf{Fact 2 (Syntactic):} English kinship terms permit bare singular use where other count nouns don't:

\ea\label{ex:baresingular}
    \ea[]{{\mention{Mom is here.}} \hfill (grammatical)}
    \ex[?]{{\mention{Teacher is here.}} \hfill (marginal without context)}
    \ex[*]{{\mention{Neighbor is here.}} \hfill (ungrammatical)}
    \z
\z

When used this way, \mention{Mom} functions semantically as a \term{name}~-- picking out a unique, identifiable individual. This semantic function is borne out syntactically: \mention{Mom} appears as a bare proper noun without a determiner, the hallmark of what \textcite{Huddleston2002} call a \term{strong proper name}. With a determiner~-- \mention{my mom}, \mention{the mom in question}~-- the same lexical item functions as a common noun with relational meaning. The bare singular pattern is remarkable because English otherwise categorically rejects bare singular count nouns in argument position (\ungram{\mention{I bought table}}).

\textbf{The gap:} Nobody has connected these facts. The sociolinguistics literature documents the address asymmetry through power/solidarity frameworks. The syntax literature documents the determiner alternation and bare singular behavior. Neither cites the other; neither proposes a causal mechanism.

\subsection{The proposed mechanism}

This paper fills that gap. The argument is simple: the social prohibition on first-name use forces high-frequency use of kinship terms in vocative and address position. That frequency~-- sustained across childhood and into adulthood~-- produces exactly the conditions under which grammaticalization toward proper-noun-like syntax occurs.

The causal pathway has four steps:

First, \emph{social prohibition}: children can't use parents' first names. The taboo is enforced by correction, by awkwardness, by the visceral wrongness of hearing a peer say \enquote{Hey, Sarah} to their mother.

Second, \emph{address function}: the prohibition forces kinship terms into the slot normally occupied by names. \mention{Mom} and \mention{Dad} become the primary vocative forms~-- the words a child uses to get a parent's attention, to address them directly, to refer to them in third-person conversation with other family members.

Third, \emph{high frequency}: this isn't occasional use. Children say \mention{Mom} and \mention{Dad} hundreds of thousands of times over the course of childhood. Each use is in a slot prototypically occupied by proper names.

Fourth, \emph{grammaticalization}: repetition in a name-like slot produces entrenchment. The form gets stored as an unanalyzed chunk with the distributional privileges of a proper noun~-- it appears without a determiner, it resists modification in ways that common nouns don't~-- while acquiring the semantic function of a name: unique reference to a specific individual.

Grammar, on this view, is \enquote{sedimented behavior} \citep{bybee2010}, not a set of abstract rules that \enquote{allow} exceptions. \mention{Mom} isn't a common noun that the grammar happens to permit in a bare-singular slot. It's a form that has acquired proper-name syntax through the pressure of its use.

\subsection{The CGEL framework}

The analytical apparatus comes from \emph{The Cambridge Grammar of the English Language} \citep{Huddleston2002}, which maintains strict category--function distinctions that earlier treatments conflate.

\begin{table}[h]
\centering
\begin{tabular}{lll}
\toprule
Concept & Definition & Example \\
\midrule
Proper noun & Lexical category (single word) & \mention{John}, \mention{Mom} \\
Proper name & NP functioning as name & \mention{the United Kingdom} \\
Strong proper name & Used without article & \mention{Paris}, \mention{Mom} \\
Weak proper name & Requires \mention{the} & \mention{the Thames} \\
Bare singular & Count noun without determiner & Normally ungrammatical \\
\bottomrule
\end{tabular}
\caption{Key terminology from CGEL}
\label{tab:cgel}
\end{table}

Kinship terms show a functional split:

\ea\label{ex:split}
    \ea Semantically as a name, syntactically as a bare proper noun: \mention{Mom called.}
    \ex Semantically relational, syntactically as a common noun with determiner: \mention{My mom called.}
    \z
\z

CGEL itself notes this split in its treatment of vocatives (\S5.20.5): kin terms are listed alongside personal names as NPs that can serve vocative function, and the text explicitly observes that \enquote{the kin terms include those that can be used with the status of proper names} \citep[522]{Huddleston2002}. \mention{Mum}, \mention{Mom}, and \mention{Mummy} are given as examples. Other kin terms~-- \mention{son}, \mention{cousin}~-- are \enquote{hardly possible as proper names,} showing that the category is graded rather than uniform.

This isn't polysemy in the usual sense. It's positional: the same lexical item acquires different syntactic privileges depending on whether it's functioning semantically as a name or as a descriptive common noun. The grammaticalization story explains \emph{why} kinship terms developed this split: the taboo forces them into name function so frequently that they've acquired the syntactic distributional properties of proper nouns.

\subsection{Relation to deitality and definiteness}

In a companion analysis, I distinguish \term{deitality}~-- a morphosyntactic property cluster~-- from \term{definiteness}~-- a semantic one. Deital determiners (\mention{the}, \mention{this}, \mention{my}) show distributional restrictions: they resist existential pivots, license partitive complements, host identificational constructions. Definiteness is interpretive: identifiability, uniqueness, anaphoric recoverability.

Proper names occupy a distinctive position in this framework: they're \emph{semantically definite} (they refer to unique, identifiable individuals) but \emph{morphosyntactically non-deital} (they appear without determiners and don't show the distributional restrictions of deital NPs). English kinship terms, when functioning as names, pattern exactly this way. They're definite in meaning but bare in form~-- grammaticalized fossils of the taboo that shaped their use.

This explains why \mention{Mom called} behaves more like \mention{Paris welcomed visitors} than like \mention{The mother called}. Both \mention{Mom} and \mention{Paris} are strong proper names: definite in interpretation, non-deital in morphosyntax. The kinship term has been pulled toward proper-name status by the social pressure of address.

\subsection{What follows}

The rest of the paper proceeds as follows. Section~\ref{sec:socio} reviews the sociolinguistic literature on address forms and the naming taboo. Section~\ref{sec:syntax} reviews the syntactic literature on proper nouns, bare nominals, and the CGEL apparatus. Section~\ref{sec:mechanism} presents the core argument: the grammaticalization pathway from taboo to syntax. Section~\ref{sec:crossling} addresses the cross-linguistic dimension. Section~\ref{sec:objections} considers objections and alternative explanations. Section~\ref{sec:conclusion} concludes.

%% PLACEHOLDER SECTIONS %%

\section{The naming taboo}\label{sec:socio}

[To be drafted: Brown \& Gilman 1960, Brown \& Ford 1961, Dickey 1997, Braun 1988]

\section{Kinship terms as proper nouns}\label{sec:syntax}

[To be drafted: CGEL apparatus, bare singulars, Hill 2022]

\section{From taboo to syntax}\label{sec:mechanism}

[To be drafted: Bybee, Traugott, grammaticalization pathway]

\section{Cross-linguistic perspective}\label{sec:crossling}

[To be drafted: English taboo is \enquote{weak} compared to Chinese, Vietnamese]

\section{Objections and alternatives}\label{sec:objections}

[To be drafted]

\section{Conclusion}\label{sec:conclusion}

[To be drafted]

\printbibliography

\end{document}
